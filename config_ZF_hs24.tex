\usepackage{graphicx}
\usepackage{titlesec}
\usepackage{amsmath}
\usepackage[german]{datetime}
\usepackage{float}
\usepackage{caption}
\usepackage{fancyhdr}
\usepackage{lastpage}
\usepackage{kantlipsum}
\usepackage[ngerman]{babel}
\usepackage[T1]{fontenc}
\usepackage[utf8]{inputenc}
\usepackage{chngcntr}
\usepackage{tocloft}
\usepackage{titlepic}
\usepackage{array}
\usepackage{setspace}
\usepackage{url}
\usepackage[colorlinks=false]{hyperref}
\usepackage{tabularx}
\usepackage{multicol}
\usepackage{listings}
\usepackage[outputdir=output]{minted}
\usepackage{tcolorbox}
\usepackage{xcolor}
\usepackage{amssymb}
\usepackage{mathrsfs}
\usepackage{trsym,trfsigns}
\usepackage{enumitem}
\usepackage{xcolor}

% In die Präambel (vor \begin{document}):

\renewcommand{\arraystretch}{1.3}

\usepackage[T1]{fontenc}
\usepackage[utf8]{inputenc} % falls du kein lualatex/xelatex nutzt
\usepackage{xcolor}
\usepackage{listings}

% Structured Text (IEC 61131-3) Sprache definieren
\lstdefinelanguage{StructuredText}{
morekeywords=[1]{
PROGRAM, FUNCTION, FUNCTION_BLOCK, VAR, VAR_INPUT, VAR_OUTPUT, VAR_IN_OUT,
VAR_TEMP, VAR_GLOBAL, VAR_EXTERNAL, END_VAR, END_PROGRAM, END_FUNCTION,
END_FUNCTION_BLOCK, IF, THEN, ELSIF, ELSE, END_IF, CASE, OF, END_CASE,
FOR, TO, BY, DO, END_FOR, WHILE, END_WHILE, REPEAT, UNTIL, END_REPEAT,
RETURN, EXIT, WITH, CONSTANT, RETAIN, AT,
TRUE, FALSE, AND, OR, XOR, NOT, MOD
},
morekeywords=[2]{BOOL, BYTE, WORD, DWORD, LWORD, SINT, INT, DINT, LINT, USINT, UINT, UDINT, ULINT, REAL, LREAL, TIME, DATE, TOD, DT, STRING, WSTRING},
sensitive=false,
morecomment=[l]{//},
morecomment=[s]{(*}{*)},
morestring=[b]",
}

% Globaler Style
\lstdefinestyle{st}{
language=StructuredText,
basicstyle=\ttfamily\small,
keywordstyle=[1]\color{blue!70!black}\bfseries,
keywordstyle=[2]\color{teal!70!black},
commentstyle=\color{gray!70!black}\itshape,
stringstyle=\color{orange!70!black},
numbers=left,
numberstyle=\tiny\color{gray},
stepnumber=1,
numbersep=8pt,
showstringspaces=false,
tabsize=2,
breaklines=true,
breakatwhitespace=true,
xleftmargin=1.5em,
xrightmargin=0.5em,
frame=single,
framerule=0.4pt,
rulecolor=\color{black!20},
captionpos=b,
keepspaces=true,
columns=fullflexible,
% für Umlaute im Code:
literate=
{ä}{{"a}}1 {ö}{{"o}}1 {ü}{{"u}}1
{Ä}{{"A}}1 {Ö}{{"O}}1 {Ü}{{"U}}1
{ß}{{\ss}}1
}

% Optional: ein Shortcut
\newcommand{\stlisting}{\lstset{style=st}}


% Definition der Rot töne
\definecolor{DarkRed}{HTML}{8B0000}
\definecolor{LightRed}{HTML}{FFDDDD}

% Definition der Orange-Töne
\definecolor{VeryLightOrange}{HTML}{F0F0F0}
\definecolor{LightOrange}{HTML}{D9E9CF}
\definecolor{MediumOrange}{HTML}{B6CEB4}
\definecolor{DarkOrange}{HTML}{96A78D}

% Definition der Blau-Töne
\definecolor{LightBlue}{HTML}{EAF7FF}
\definecolor{DarkBlue}{HTML}{005F9E}

%-----Seitenränder-----%
\usepackage[a4paper, left=1cm, right=1cm, top=2cm, bottom=2cm]{geometry}

%-----Spalteneinstellungen-----%
\setlength{\columnsep}{20pt}
\setlength{\columnseprule}{0.5pt} % Dicke der vertikalen Linie


%-----Titelformatierungen-----%
\titleformat{\chapter}{\sffamily\Large\bf\sffamily\rlap{\color{DarkOrange}\rule[-0.5ex]{1\linewidth}{3ex}\vspace{-3ex}}\sffamily\Large\bf\sffamily\color{black}}{\thechapter}{15pt}{}
\titleformat{\section}{\sffamily\large\bf\sffamily\rlap{\color{MediumOrange}\rule[-0.5ex]{\linewidth}{3ex}\vspace{-2.8ex}}\sffamily\large\bf\sffamily\color{black}}{\thesection}{3pt}{}
\titleformat{\subsection}{\sffamily\rlap{\color{LightOrange}\rule[-0.5ex]{\linewidth}{3ex}\vspace{-2.6ex}}\sffamily\sffamily\color{black}}{\thesubsection}{3pt}{}
\titleformat{\subsubsection}{\sffamily\rlap{\color{blue!10!white}\rule[-0.5ex]{\linewidth}{3ex}\vspace{-3ex}}\sffamily\sffamily\color{black}}{\thesubsection}{3pt}{}

%\titleformat{\subsubsection}{\sffamily}{\thesubsection}{10pt}{}
\titlespacing*{\chapter}{5pt}{5pt}{5pt}
\titlespacing*{\section}{5pt}{5pt}{5pt}
\titlespacing*{\subsection}{5pt}{5pt}{5pt}
\renewcommand{\cfttoctitlefont}{\Huge\bf\sffamily}
\renewcommand{\cftloftitlefont}{\sffamily\section}
\renewcommand{\cftlottitlefont}{\sffamily\section}
\addtocontents{lof}{\protect\vspace{-3\baselineskip}}
\addtocontents{lot}{\protect\vspace{-3\baselineskip}}
\addtocontents{toc}{\protect\vspace{-3\baselineskip}}
\captionsetup{font={sf}}
\renewcommand{\normalsize}{\fontsize{8}{8}\selectfont}
\titleclass{\chapter}{straight}

%-----Kopf- und Fusszeile-----%
\pagestyle{fancy}
\fancyhf{}
\lhead{\today}
\chead{}
\rhead{IAUT - Eric Lindegger}
\renewcommand{\headrulewidth}{0.1pt}
\lfoot{HSLU T\&A}
\cfoot{}
\rfoot{\thepage}
\renewcommand{\footrulewidth}{0.1pt}
%Kopf- und Fusszeile erzwingen überall
\patchcmd{\chapter}{\thispagestyle{plain}}{\thispagestyle{fancy}}{}{}

%-----Zähler-----%
\newcounter{simplecount}
\setcounter{simplecount}{0}
\renewcommand{\theequation}{\arabic{simplecount}}
\newcommand{\owncount}{\refstepcounter{simplecount}}

\newcounter{simplecount2}
\setcounter{simplecount2}{0}
\renewcommand{\thetable}{\arabic{simplecount2}}
\newcommand{\tabularcount}{\refstepcounter{simplecount2}}

\newcounter{simplecount3}
\setcounter{simplecount3}{0}
\renewcommand{\thefigure}{\arabic{simplecount3}}
\newcommand{\figurecount}{\refstepcounter{simplecount3}}

%-----Abbildungsbeschriftung-----%
\counterwithout{figure}{chapter}

\renewcommand{\cftfigpresnum}{Abbildung }
\renewcommand{\cfttabpresnum}{Tabelle }

\renewcommand{\cftfigaftersnum}{:}
\renewcommand{\cfttabaftersnum}{:}

\setlength{\cftfignumwidth}{2.5cm}
\setlength{\cfttabnumwidth}{2.5cm}

\setlength{\cftfigindent}{0cm}
\setlength{\cfttabindent}{0cm}

\renewcommand{\figurename}{Abbbildung}
\renewcommand{\tablename}{Tabelle}

\setlength{\parindent}{0pt}